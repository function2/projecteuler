% 2014 Michael Seyfert <michael@codesand.org>
\documentclass[12pt]{article}
\usepackage{natbib}
\usepackage{makeidx}
\input{/home/michael/my/general.tex}

\usepackage{tikz}
\usetikzlibrary{decorations,decorations.markings}

%%%%%%%%%%%%%%%%%%%%%%%%%%%%%%%%%%%%%%%%%%%%%%%%%%%%%%%%%%%%%%%%%%%%%%%%%%%%%%%%
% Page Setup.
%%%%%%%%%%%%%%%%%%%%%%%%%%%%%%%%%%%%%%%%%%%%%%%%%%%%%%%%%%%%%%%%%%%%%%%%%%%%%%%%
\setmainfont[Numbers=Uppercase]{Fanwood}
\setmathsfont{Minion Pro}
\usepackage{setspace}
\linespread{1.2}
%%%%%%%%%%%%%%%%%%%%%%%%%%%%%%%%%%%%%%%%%%%%%%%%%%%%%%%%%%%%%%%%%%%%%%%%%%%%%%%%

% \numberwithin{equation}{section}

%\makeindex

%%%%%%%%%%%%%%%%%%%%%%%%%%%%%%%%%%%%%%%%%%%%%%%%%%%%%%%%%%%%%%%%%%%%%%%%%%%%%%%%
%% Colors
%%%%%%%%%%%%%%%%%%%%%%%%%%%%%%%%%%%%%%%%%%%%%%%%%%%%%%%%%%%%%%%%%%%%%%%%%%%%%%%%
\hypersetup{
    colorlinks,%
    citecolor=black,%
    filecolor=black,%
    linkcolor=black,%
     urlcolor=black,
    pdfauthor={Michael Seyfert <\myemail>},
    pdfsubject={Project Euler},
    pdftitle={Project Euler Solutions},
    pdfkeywords={mathematics}
}

\definecolor[named]{section-color}{rgb}{.11,.23,.7}
\definecolor[named]{medium-blue}{rgb}{.11,.23,.8}

%% \def\mhighlight#1{
%%   \colorbox{lightblue}{\boxed{#1}}
%% }

\def\mhighlight#1{
{\boxed{#1}}
}

%colored titles
\def\mysubsection#1{
\subsection{\textcolor{section-color}{#1}}
}

\def\mysubsubsection#1{
\subsubsection{\textcolor{section-color}{#1}}
}

\def\mhighlight#1{
\colorbox{lb}{\boxed{#1}}
}

% Colorize
\newcommand{\colize}[1]{\color{medium-blue}#1\color{black}}
%%%%%%%%%%%%%%%%%%%%%%%%%%%%%%%%%%%%%%%%%%%%%%%%%%%%%%%%%%%%%%%%%%%%%%%%%%%%%%%%
%% Title pages.
%%%%%%%%%%%%%%%%%%%%%%%%%%%%%%%%%%%%%%%%%%%%%%%%%%%%%%%%%%%%%%%%%%%%%%%%%%%%%%%%
\begin{document}
\title{Project Euler Solutions}
\author{Michael Seyfert <\href{mailto:\myemail}{\myemail}>}
\maketitle
Solutions to problems on \url{https://projecteuler.net}

\begin{itemize}
\colize{\item} Problem 28.
\[
A =
\begin{bmatrix}
21 & 22 & 23 & 24 & 25 \\
20 &  7 &  8 &  9 & 10 \\
19 &  6 &  1 &  2 & 11 \\
18 &  5 &  4 &  3 & 12 \\
17 & 16 & 15 & 14 & 13 \\
\end{bmatrix}
\qquad \text{for} \quad n =5
\]
\begin{align*}
  \text{Upper right diagonals}&\qquad \sum_{1 \le k \le \ceil{n/2}} (2k-1)^2 \\
%  \text{Lower left diagonals}&\qquad \sum_{1 \le k \le \ceil{n/2}} \floor{\frac{(2k-1)^2}{2}} + (2k-1).
\end{align*}
Closed form is
\[
1 + \overbrace{(3 + 5 + 7 + 9)}^{2\text{ diff}} + \overbrace{(13 + 17 + 21 + 25)}^{4\text{ diff}} = 101.
\]

\colize{\item} Problem 29.

\[
a^b \qquad\text{for}\quad 2 \le a \le 5,\qquad 2 \le b \le 5.
\]
How many distinct terms are in $a^b$ for $2 \le a,b \le 100$?

\colize{\item} Problem 30.
\[
\text{number of digits in $x$ (base $b$)} = d_b(x) = \lceil\log_b(x+1)\rceil \quad \text{integer } x \ge 1
\]
Find the smallest $n$ such that
\[10^n - 1 > n(9^5)\]
This gives $n = 6$. This means there are at most $6$ digits in a number
that satisfies the property we want.

\end{itemize}
%%%%%%%%%%%%%%%%%%%%%%%%%%%%%%%%%%%%%%%%%%%%%%%%%%%%%%%%%%%%%%%%%%%%%%%%%%%%%%%%
% Bibliography
%%%%%%%%%%%%%%%%%%%%%%%%%%%%%%%%%%%%%%%%%%%%%%%%%%%%%%%%%%%%%%%%%%%%%%%%%%%%%%%%
%% \bibliographystyle{plainnat}
%% \bibliography{pe_bib}
%%%%%%%%%%%%%%%%%%%%%%%%%%%%%%%%%%%%%%%%%%%%%%%%%%%%%%%%%%%%%%%%%%%%%%%%%%%%%%%%

% Use 'makeindex projecteuler' command to make the index.
%\printindex

\end{document}
